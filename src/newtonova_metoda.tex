\subsection{Newtonova metoda (metoda tangente)}

Newtonova metoda se smatra standardnom metodom rješavanja nelinearnih jednadžbi
oblika $f(x)=0$, uz pretpostavku da je $f$ diferencijabilna te da je $f'(x) \neq
0$ u nekoj okolini nultočke $\xi$.

Neka je $x_0$ zadana početna aproksimacija tražene nultočke $\xi$. Prema
Taylorovom teoremu funkcija $f$ se može aproksimirati Taylorovim polinomom:
$$
f(x_1) = f(x_0) + f'(x_0)(x_1-x_0)+\dfrac{f''(\xi)}{2}(x_1-x_0)^2,
$$
pri čemu se $\xi$ nalazi između $x_0$ i $x_1$. Zanemarimo li kvadratni član,
dobivamo:
$$
f(x_1) \approx f(x_0) + f'(x_0)(x_1-x_0).
$$

Ako je $x_1$ dovoljno blizu nultočke $\xi$, onda je $f(x_1)\approx 0$ pa je
$$
0 \approx f(x_0) + f'(x_0)(x_1-x_0).
$$

Dakle, sljedeću aproksimaciju biramo tako da vrijedi:
$$
x_1 = x_0 - \dfrac{f(x_0)}{f'(x_0)}.
$$


Generalizacijm dolazimo do sljedeće sheme:
\begin{equation}
    \label{eq:metoda_tangente}
    x_{n+1} = x_n - \dfrac{f(x_n)}{f'(x_n)},\quad n \in \mathbb{N}_0.
\end{equation}

\fancyeq[Relacijom]{eq:metoda_tangente} definirana je metoda tangente (Newtonova
metoda) za traženje nultočaka funkcije $f$. Kao i kod svake druge iterativne
metode, potrebno je zadati početnu aproksimaciju $x_0$. Svaka sljedeća iteracija
računa se prema prethodnoj iteraciji dobivenoj prema
\fancyeq[formuli]{eq:metoda_tangente}.

Nedostaci metode tangente su to što:
\begin{itemize}
    \item nije moguče garantirati konvergenciju,
    \item derivacija $f'(x)$ se treba eksplicitno izračunati u svakoj točki
    aproksimacije,
    \item početni korak mora biti dovoljno blizu rješenja.
\end{itemize}

% TODO: Slika newtonove metode

\subsubsection{Konvergencija}

Neka funkcija $f: [a,b] \to \mathbb{R}$ ima neprekidnu drugu derivaciju na
segmentu $[a,b]$. Neka je $f(a)f(b) < 0$, odnosno neka funkcija $f$ mijenja
predznak na tom segmentu.
Neka prva i druga derivacije funkcije $f$ imaju stalan predznak na promatranom
segmentu.

Ako je početna aproksimacija odabrana tako da vrijedi
$$
    f(x_0)f''(x_0) > 0
$$
tada niz aproksimacija dobiven metodom tangente konvergira prema jedinstvenom
rješenju jednadžbe $f(x) = 0$.

\subsubsection{Ocjena greške}

Za aproksimaciju metodom tangente vrijedi ocjena greške
$$
|\xi - x_i| \leq \dfrac{M_2}{2m_1}(x-x_{i-1})^2,
$$
gdje je $M_2 = \max_{x\in [a,b]}|f''(x)|$ i $m_1 = \min_{x\in [a,b]}|f'(x)|$.

Metoda ima kvadratnu brzinu konvergencije:
$$
|\xi - x_{x+1}| \leq \dfrac{M_2}{2m_1}(x-x_i)^2.
$$

\subsubsection{Algoritam}

\begin{algorithmic}
\Function{Newton}{$x,\;\varepsilon,\;f,\;f',\;f'',\;n$}
    \If{$f(x) \cdot f''(x) < 0$}
        \Error{Newtonova metoda ne konvergira.\linebreak Treba izabrati drugu početnu aproksimaciju.}
    \Else
        \State $i \gets 1$
        \While{$i \leq n$}
            \If{$\lvert f'(x) \rvert < \varepsilon$}
                \State \textbf{break} \Comment{Horizontalna tangenta}
            \EndIf
            \State $dx \gets \dfrac{f(x)}{f'(x)}$
            \State $x \gets x - dx$
            \If{$\lvert dx \rvert < \varepsilon$}
                \State \Return $x,\;i$
            \Else
                \State $i \gets i + 1$
            \EndIf
        \EndWhile
        \Error{Rješenje na zadanu točnost nije nađeno u zadanom broju iteracija.}
    \EndIf
\EndFunction
\end{algorithmic}
