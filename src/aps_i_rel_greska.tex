\section{Apsolutna i relativna greška}

Razliku $a-a^*$ između \textit{stvarne veličine} $a$ i \textit{njene aproksimacije} $a^*$ nazivamo \textbf{greška aproksimacije}. Apsolutnu vrijednost greške aproksimacije nazivamo \textbf{apsolutna greška aproksimacije} i označavamo je s $\Delta a^*$. Vrijedi:

$$
\Delta a^* = |a-a^*|
$$

Omjer između apsolutne greške $\Delta a^*$ i apsolutne vrijednosti $|a|$ nazivamo \textbf{relativna greška} i označavamo je s $\delta a^*$. Vrijedi:

$$
\delta a^* = \frac{\Delta a^*}{|a|},\qquad a\neq0
$$

S obzirom na to da u praksi često nije poznata točna vrijednost $a$, koristi se \textbf{aproksimacija relativne greške}:

$$
\delta a^* \approx \frac{\Delta a^*}{|a^*|},\qquad a\neq0
$$

\begin{itemize}
    \item Redoslijed izvršavanja operacija na računalu je bitan jer utječe na veličinu greške zbog ograničene pohrane decimalnih brojeva (floating point error).
    \begin{itemize}
        \item Zbrajanje i množenje nisu asocijativni
        \item Množenje prema zbrajanju nije distributivno
    \end{itemize}
\end{itemize}

\subsection{Katastrofalno kraćenje}

Oduzimanje brojeva $x$ i $y$ koji su istog predznaka može izazvati proizvoljno velike greške kada je $|x-y| \ll |x|,|y|$

Za katastrofalno kraćenje vrijedi:

\begin{gather*}
(x-y)(1+\varepsilon) = x(1+\varepsilon_x) - y(1+\varepsilon_y)\\
|\varepsilon| \leq \left|\frac{x}{x-y}\right| |\varepsilon_x| + \left|\frac{y}{x-y}\right| |\varepsilon_y|
\end{gather*}

Jedna od metoda izbjegavanja kraćenja je racionalizacija korijenu koja daje puno točniji rezultat za $x_1$:
$$
    x_1=-p+\sqrt{p^2+q}\cdot\frac{-p-\sqrt{p^2+q}}{-p-\sqrt{p^2+2}}=\frac{q}{p+\sqrt{p^2+q}}
$$
