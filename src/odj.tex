\textbf{Cauchyjev problem:} traži rješenje parcijalne diferencijalne jednadžbe
koje zadovoljava zadane uvjete za diferenciranu jednadžbu.

Cilj je odrediti rješenje Cauchyjevog problema, tj. integralnu krivulju koja
prolazi zadanom tokom $M(x_0,y_0)$. Ako rješenje postoji, ono je jedinstveno
\footnote{Picardov teorem uz Lipschitzov uvjet}.

\section{Eulerova metoda}

Eulerova metoda je inspirirana geometrijskom interpretacijom derivacije, odnosno
vezom derivacije i tangente na krivulju.

Funkciju lokalno aproksimiramo tangentom:
$$
y_{n+1} = y_n + h\cdot f(x_n, y_n)
$$

Gdje je $f(x_n, y_n)$ diferencijalna jednadžba koja predstavlja derivaciju $y$ u
odnosu na $x$ tj. njena vrijednost je $\frac{dy}{dx}$. A $h$ je veličina
eulerovog koraka.

Za mali korak $h$ točka $T_1(x_1, y_1) = (x_0 + h, y_0 + h \cdot f(x_0, y_0))$
će biti dobra aproksimacija prave vrijednosti $f(x_1)$. Točku $T_1$ se zatim
promatra kao polaznu točku za sljedeću aproksimaciju, i tako do $n$
aproksimiranih segmenata.

Za neki promatrani interval $[x_0, d]$, za koji želimo aproksimirati $n$ koraka
vrijedi:
\begin{gather*}
    h=\frac{d - x_0}{n},\\
    x_i = x_{i-1} + h,\quad i \in [1, n],\\
    y_i = y_{i-1} + h\cdot f(x_{i-1}, y_{i-1}),\quad i \in [1, n].
\end{gather*}

Nedostatak ove metode je što se greška globalno akumulira i što ako su koraci
veliki, krajnji rezultat može jako odstupati od stvarnog. Kako bi dobili bolje
rezultate, korak $h$ mora biti izrazito mali.
\smallskip
\textbf{Lokalna greška} ove metode je reda 2, no globalno se akumulira pa je
metoda samo prvog reda točnosti. Ovisno o kontekstu promatranja (npr. kaotično
kretanje čestica), globalna ocjena greška može iznimno odstupati od stvarne
greške.

\begin{multicols}{2}
\subsection{Algoritam eulerove metode}

\begin{algorithmic}
    \Function{Euler}{$x_0,\;y_0,\;h,\;f,\;n$}
        \State $x \gets [x_0]$
        \State $y \gets [y_0]$
        \For{$i \in [0, n\rangle$}
            \State $slope \gets f(x[i], y[i])$
            \State $x_{\text{new}} \gets x[i] + h$
            \State $x \gets [...\,x,\;x_{\text{new}}]$
            \State $y_{\text{new}} \gets y[i] + h \cdot slope$
            \State $y \gets [...\,y,\;y_{\text{new}}]$
        \EndFor
        \State \Return $x,\;y$
    \EndFunction
\end{algorithmic}

\columnbreak

\subsection{Eulerova metoda u Pythonu}
\input{../code/euler}

\end{multicols}

\section{Runge-Kutta metode}

Ideja je koristiti Taylorov polinom nekog višeg stupnja $m$:

$$
y_{i+1} = y_i + h\cdot y'(x_i) + h^2 \cdot y''(x_i) + \dots + h^m \cdot y^{(m)}(x_i)
$$

umjesto tangente, tj. Taylorovog polinoma prvog stupnja, kao što Eulerova metoda
koristi. Tada je lokalna greška reda $m + 1$. No za to je potrebno izračunati i
više derivacije funkcije, što ne možemo dobiti samo iz diferencijalne jednadžbe
prvog stupnja.

Runge-Kutta metode temelje se na ideji da se umjesto direktne interpolacije za
korak $h$ i koeficijent smjera $k_1 = f(x_i,y_i)$ u nekoj točki, koristi
linearna kombinacija koeficijenata smjera $k_1$ i $k_2$, gdje je $k_2$
koeficijent smjera dobiven za iterpolaciju u točki $x_i + p_1\cdot h$.

\begin{align*}
    k_1 &= f(x_i,y_i)\\
    k_2 &= f(x_i + h \cdot p_1,h\cdot k_1\cdot p_1)\\
    y_{i+1} &= y_i + h (a_1\cdot k_1 + a_2 \cdot k_2)
\end{align*}

Pomak $y_{i+1}$ je jednak Taylorovom polinomu drugog stupnja ako uzmemo
vrijednosti:
$$
a_1 + a_2 = 1,\quad a_2 p_1 = \frac{1}{2}
$$

No, ovisno o odabiru parametara $p_1$, $a_1$ i $a_2$, dobivamo različite
numeričke metode. Neke od Runge-Kutta metoda drugog reda su:

\begin{center}
\begin{tabular}{r|c|c|c}
    & $a_1$ & $a_2$ & $p_1$\\
    \hline
    Heuneova metoda & $1/2$ & $1/2$ & $1$\\
    \hline
    Popravljena poligonalna metoda & $0$ & $1$ & $1/2$\\
    \hline
    Ralstonova metoda & $1/3$ & $2/3$ & $3/4$\\
\end{tabular}
\end{center}

Slično, ali s linearnom kombinacijom od tri koeficijenta smjera u tri točke
dobivamo \textbf{lokalnu točnost} kao da smo koristili Taylorov polinom trećeg
stupnja i time konstruiramo različite Runge-Kutta metode trećeg reda točnosti
globalno.

Tipično se odabire Runge-Kutta metoda četvrtog reda, jer daljnjim povećanjem
reda točnosti metode se znatno smanjuje stabilnost i povećava zahtjevnost
proračuna, pa isplativost postaje upitna.

Obično se isplati koristiti Runge-Kutta metode više od eulerove jer unatoč većem
broju potrebnih izračuna, te metode daju puno bolje rezultate za isti korak $h$
od eulerove pa onda i korak može biti veći.

Generalizacija \textit{eksplicitnih} Runge-Kutta metoda, za $s$-ti red točnosti je:

\begin{align*}
    k_1 & = f(t_n, y_n),\\
    k_2 & = f(t_n+c_2h, y_n+(a_{21}k_1)h),\\
    k_3 & = f(t_n+c_3h, y_n+(a_{31}k_1+a_{32}k_2)h),\\
        &\quad \vdots\\
    k_s & = f(t_n+c_sh, y_n+(a_{s1}k_1+a_{s2}k_2+\cdots+a_{s,s-1}k_{s-1})h),\\
    y_{n+1} &= y_n + h \sum_{i=1}^s b_i k_i.
\end{align*}

Zbog nestabilnosti eksplicitnih Runge-Kutta metoda se pojavljuju i implicitne,
koje su značajne za rješavanje parcijalnih diferencijalnih jednadžbi, no one su
i značajno skuplje za izračun. Primjer takve metode je Gauss-Legendre metoda.

\begin{multicols}{2}
    
\subsection{Algoritam za RK4}

\begingroup
\makeatletter
\@totalleftmargin=-0.5cm
\begin{algorithmic}
    \Function{RK4}{$x_0,\;y_0,\;h,\;f,\;n$}
        \State $x \gets [x_0]$
        \State $y \gets [y_0]$
        \For{$i \in [0, n\rangle$}
            \State $k_1 \gets f(\;x[\,i\,], \phantom{\ \,+ h \cdot \frac{1}{2}} \;y[\,i\,] \phantom{\ + h \cdot \frac{1}{2}\cdot k_1}\;)$
            \State $k_2 \gets f(\;x[\,i\,] + h \cdot \frac{1}{2},               \;y[\,i\,] + h \cdot \frac{1}{2}\cdot k_1\;)$
            \State $k_3 \gets f(\;x[\,i\,] + h \cdot \frac{1}{2},               \;y[\,i\,] + h \cdot \frac{1}{2}\cdot k_2\;)$
            \State $k_4 \gets f(\;x[\,i\,] + h, \phantom{\ \,\cdot \frac{1}{2}} \;y[\,i\,] + h \cdot k_3 \phantom{\ \frac{1}{2}\cdot}\;)$
            
            \State $x_{\text{new}} \gets x[\,i\,] + h$
            \State $x \gets [...\,x, x_{\text{new}}]$
            
            \State $y_{\text{new}} \gets y[\,i\,] + h \cdot \frac{1}{6} \cdot (k_1 + 2k_2 + 2k_3 + k_4)$
            \State $y \gets [...\,y, y_{\text{new}}]$
        \EndFor
        \State \Return $x,\;y$
    \EndFunction
\end{algorithmic}
\endgroup

\columnbreak

\subsection{Implementacija RK4 u Pythonu}

\begingroup
\makeatletter
\@totalleftmargin=-1cm
\input{../code/runge_kutta}

\endgroup

\end{multicols}

\newpage

\section{Sustavi običnih diferencijalnih jednadžbi}

Sustav $n$ diferencijalnih jednadžbi prvog reda može se zapisati u obliku jedne
diferencijalne jednadžbe $n$-tog reda. Također, diferencijalnu jednadžbu $n$-tog
reda može se zapisati u obliku sustava $n$ diferencijalnih jednadžbi prvog reda.

Vrijedi:

$$
\dfrac{d^{(n)}y}{dx^{(n)}} = f(x, y, y', \dots, y^{(n-1)}),
$$

za što uvodimo supstituciju

$$
    z_i = y^{(i-1)},\quad i \in [1, n].
$$

\textbf{Primjer:}

Zadana je jednadžba

$$
2y''-5y'+y=0
$$

uz početne uvjete $y(3)=6$ i $y'(3)=-1$. Uvodimo novu varijablu $z$ i definiramo
da je $z_1=y$ i $z_2=y'$. U tom slučaju je

\begin{gather*}
    2y''=5z_2 - z_1,\\
    y''=-\frac{1}{2}z_1+\frac{5}{2}z_2.
\end{gather*}
