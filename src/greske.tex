\section{Greške}

Greške su neizbježne, prilikom izračuna je cilj smanjiti njihov utjecaj na krajnji rezultat.

\subsection{Vrste grešaka}

\begin{itemize}
    \item \textbf{Greška modela} nastaje zamjenom nekog složenog sustava ili procesa jednostavnijim.
    \begin{itemize}
        \item Neki sustavi su pre složeni da bi se opisali i riješili poznatim matematičkim alatima.
        \item Primjer: otpor zraka u balistici
    \end{itemize}
    \item \textbf{Greška metode} nastaje zamjenom procesa (obično beskonačnog) konačnom aproksimacijom.
    \begin{itemize}
        \item Primjer: određivanje sume beskonačnog reda
    \end{itemize}
    \item \textbf{Greška diskretizacije} nastaje kada umjesto podataka koje nije moguće potpuno ispravno pohraniti ili predstaviti koristimo njima bliske podatke (aproksimacije).
    \begin{itemize}
        \item Primjeri:
        \begin{itemize}
            \item $\pi, e, \dots$
            \item Zamjena kontinuuma konačnim diskretnim skupom
            \item Zamjena beskonačno male veličine s malom, konkretnom veličinom
        \end{itemize}
    \end{itemize}
    \item \textbf{Ulazna pogreška} nastaje zbog pogrešaka u ulaznim podacima nastalih prethodnom obradom podataka ili greškom mjerenja.
    \begin{itemize}
        \item Primjer: akceleracija sile teže.
    \end{itemize}
\end{itemize}

Zaokruživanjem ili odbacivanjem decimala u prikazu broja a u računalu dobivamo aproksimaciju realnog broja (\textit{engl.} floating point approximation) od $a$ koju označavamo s fl$(a)$ i za koju vrijedi:

$$
\text{fl}(a) = a(1+\varepsilon)
$$

gdje je $\varepsilon$ \textbf{greška aproksimacija}.

\newpage

\subsection{Zapis realnog broja u računalu}

Za zapis realnih brojeva se generalno koristi IEEE-754 standard.

U IEEE-754 standardu su brojevi pohranjeni s tri komponente:

\begin{center}
    \begin{tabular}{|l|l|}
        \hline
        Ime&Opis\\
        \hline
        Predznak&\verb|1| ako se radi o negativnom broju\\
        Eksponent&eksponent na bazi pohranjenog broja\\
        Mantisa/realni dio&decimalni dio realnog broja\\
        \hline
    \end{tabular}
\end{center}

Prilikom pohrane se podrazumijeva da mantisa ima prefiksni bit \verb|1| koji se ne piše jer nema smisla pohraniti broj s vodećim nulama.

IEEE-754 navodi raspored bitova za realne brojeve različitih veličina. Predznak je uvijek veličine jednog bita, a ostali karakteristike su navedene u sljedećoj tablici:

\begin{center}
    \begin{tabular}{l|c|c|c|c|c}
        Podatak&binary16&binary32&binary64&binary128&binary256\\
        \hline
        \hline
        Eksponent&5b&8b&11b&15b&19b\\
        Mantisa&10b&23b&52b&122b&236b\\
        Ukupna veličina&16b&32b&64b&128b&256b\\
        \hline
        \hline
        $ulp^*$&$2^{-11}$&$2^{-24}$&$2^{-53}$&$2^{-123}$&$2^{-237}$\\
        \hline
        \hline
        $exp_{min}$&-14&-126&-1022&-16383&-262142\\
        $exp_{max}$&+15&+127&+1023&+16384&+262143\\
        \hline
        \hline
        $val_{min}$ (normalan)&$6.10\cdot10^{-5}$&$1.18\cdot10^{-38}$&$2.23\cdot10^{-308}$&$3.36\cdot10^{-4932}$&$2.48\cdot10^{-78913}$\\
        $val_{min}$ (subnormalan$^{**}$)&$5.96\cdot10^{-8}$&$1.40\cdot10^{-45}$&$4.94\cdot10^{-324}$&$6.48\cdot10^{-4966}$&$2.25\cdot10^{-78984}$\\
        $val_{max}$&65504&$3.40\cdot10^{38}$&$1.80\cdot10^{308}$&$1.19\cdot10^{4932}$&$1.61\cdot10^{78193}$\\
    \end{tabular}
\end{center}

\begin{itemize}[label={}]
    \item * \textbf{ulp}: jedinična strojna greška, također se zove i \textit{strojni epsilon}.
    \item ** subnormalne IEEE-754 vrijednosti su spremljene manjom decimalnom preciznošću, tj. uz veću jediničnu strojnu grešku
\end{itemize}

\subsubsection{Posebne vrijednosti}

IEEE-754 brojevi također imaju posebne vrijednosti kojima simboliziraju da se prilikom operacije dogodilo prekoračenje minimalne/maksimalne vrijednosti (\verb|+inf| i \verb|-inf|), te kada je provedena nevaljana operacija nad brojem \verb|NaN| poput dijeljenja s nulom.

\begin{center}
    \begin{tabular}{ |c|c|c|c| } 
        \hline
        Ime &
        \multicolumn{3}{|c|}{Binarni zapis}\\
        \hline
        \verb|+inf|&\verb|0|&\verb|1|\dots\verb|1|&\verb|0|\dots\verb|0|\\
        \hline
        \verb|-inf|&\verb|1|&\verb|1|\dots\verb|1|&\verb|0|\dots\verb|0|\\
        \hline
        \hline
        \verb|NaN|&\verb|_|&\verb|1|\dots\verb|1|&\verb|_|\dots\verb|_|\\
        \hline
    \end{tabular}
\end{center}

Zbog toga što \verb|NaN| može biti predstavljen s mnogo različitih binarnih kombinacija, jednačenje realnih brojeva na računalu nije smisleno rješivo. Različiti programski jezici se nose s tim problemom na različite načine:

\begin{itemize}
    \item tretiraju sve \verb|NaN| vrijednosti kao istu,
    \item tretiraju svaku (čak i one čiji binarni zapisi se podudaraju) kao različitu,
    \item bacaju grešku prilikom usporedbe,
    \item imaju koncepte \textit{djelomično jednačivih/usporedivih} vrijednosti.
\end{itemize}

\subsection{Python}

Paket u Pythonu koji dozvoljava rad s \verb|float| brojevima arbitrarne duljine je \texttt{\PY{n+nn}{decimal}}:

\section{Greške}

Greške su neizbježne, prilikom izračuna je cilj smanjiti njihov utjecaj na krajnji rezultat.

\subsection{Vrste grešaka}

\begin{itemize}
    \item \textbf{Greška modela} nastaje zamjenom nekog složenog sustava ili procesa jednostavnijim.
    \begin{itemize}
        \item Neki sustavi su pre složeni da bi se opisali i riješili poznatim matematičkim alatima.
        \item Primjer: otpor zraka u balistici
    \end{itemize}
    \item \textbf{Greška metode} nastaje zamjenom procesa (obično beskonačnog) konačnom aproksimacijom.
    \begin{itemize}
        \item Primjer: određivanje sume beskonačnog reda
    \end{itemize}
    \item \textbf{Greška diskretizacije} nastaje kada umjesto podataka koje nije moguće potpuno ispravno pohraniti ili predstaviti koristimo njima bliske podatke (aproksimacije).
    \begin{itemize}
        \item Primjeri:
        \begin{itemize}
            \item $\pi, e, \dots$
            \item Zamjena kontinuuma konačnim diskretnim skupom
            \item Zamjena beskonačno male veličine s malom, konkretnom veličinom
        \end{itemize}
    \end{itemize}
    \item \textbf{Ulazna pogreška} nastaje zbog pogrešaka u ulaznim podacima nastalih prethodnom obradom podataka ili greškom mjerenja.
    \begin{itemize}
        \item Primjer: akceleracija sile teže.
    \end{itemize}
\end{itemize}

Zaokruživanjem ili odbacivanjem decimala u prikazu broja a u računalu dobivamo aproksimaciju realnog broja (\textit{engl.} floating point approximation) od $a$ koju označavamo s fl$(a)$ i za koju vrijedi:

$$
\text{fl}(a) = a(1+\varepsilon)
$$

gdje je $\varepsilon$ \textbf{greška aproksimacija}.

\newpage

\subsection{Zapis realnog broja u računalu}

Za zapis realnih brojeva se generalno koristi IEEE-754 standard.

U IEEE-754 standardu su brojevi pohranjeni s tri komponente:

\begin{center}
    \begin{tabular}{|l|l|}
        \hline
        Ime&Opis\\
        \hline
        Predznak&\verb|1| ako se radi o negativnom broju\\
        Eksponent&eksponent na bazi pohranjenog broja\\
        Mantisa/realni dio&decimalni dio realnog broja\\
        \hline
    \end{tabular}
\end{center}

Prilikom pohrane se podrazumijeva da mantisa ima prefiksni bit \verb|1| koji se ne piše jer nema smisla pohraniti broj s vodećim nulama.

IEEE-754 navodi raspored bitova za realne brojeve različitih veličina. Predznak je uvijek veličine jednog bita, a ostali karakteristike su navedene u sljedećoj tablici:

\begin{center}
    \begin{tabular}{l|c|c|c|c|c}
        Podatak&binary16&binary32&binary64&binary128&binary256\\
        \hline
        \hline
        Eksponent&5b&8b&11b&15b&19b\\
        Mantisa&10b&23b&52b&122b&236b\\
        Ukupna veličina&16b&32b&64b&128b&256b\\
        \hline
        \hline
        $ulp^*$&$2^{-11}$&$2^{-24}$&$2^{-53}$&$2^{-123}$&$2^{-237}$\\
        \hline
        \hline
        $exp_{min}$&-14&-126&-1022&-16383&-262142\\
        $exp_{max}$&+15&+127&+1023&+16384&+262143\\
        \hline
        \hline
        $val_{min}$ (normalan)&$6.10\cdot10^{-5}$&$1.18\cdot10^{-38}$&$2.23\cdot10^{-308}$&$3.36\cdot10^{-4932}$&$2.48\cdot10^{-78913}$\\
        $val_{min}$ (subnormalan$^{**}$)&$5.96\cdot10^{-8}$&$1.40\cdot10^{-45}$&$4.94\cdot10^{-324}$&$6.48\cdot10^{-4966}$&$2.25\cdot10^{-78984}$\\
        $val_{max}$&65504&$3.40\cdot10^{38}$&$1.80\cdot10^{308}$&$1.19\cdot10^{4932}$&$1.61\cdot10^{78193}$\\
    \end{tabular}
\end{center}

\begin{itemize}[label={}]
    \item * \textbf{ulp}: jedinična strojna greška, također se zove i \textit{strojni epsilon}.
    \item ** subnormalne IEEE-754 vrijednosti su spremljene manjom decimalnom preciznošću, tj. uz veću jediničnu strojnu grešku
\end{itemize}

\subsubsection{Posebne vrijednosti}

IEEE-754 brojevi također imaju posebne vrijednosti kojima simboliziraju da se prilikom operacije dogodilo prekoračenje minimalne/maksimalne vrijednosti (\verb|+inf| i \verb|-inf|), te kada je provedena nevaljana operacija nad brojem \verb|NaN| poput dijeljenja s nulom.

\begin{center}
    \begin{tabular}{ |c|c|c|c| } 
        \hline
        Ime &
        \multicolumn{3}{|c|}{Binarni zapis}\\
        \hline
        \verb|+inf|&\verb|0|&\verb|1|\dots\verb|1|&\verb|0|\dots\verb|0|\\
        \hline
        \verb|-inf|&\verb|1|&\verb|1|\dots\verb|1|&\verb|0|\dots\verb|0|\\
        \hline
        \hline
        \verb|NaN|&\verb|_|&\verb|1|\dots\verb|1|&\verb|_|\dots\verb|_|\\
        \hline
    \end{tabular}
\end{center}

Zbog toga što \verb|NaN| može biti predstavljen s mnogo različitih binarnih kombinacija, jednačenje realnih brojeva na računalu nije smisleno rješivo. Različiti programski jezici se nose s tim problemom na različite načine:

\begin{itemize}
    \item tretiraju sve \verb|NaN| vrijednosti kao istu,
    \item tretiraju svaku (čak i one čiji binarni zapisi se podudaraju) kao različitu,
    \item bacaju grešku prilikom usporedbe,
    \item imaju koncepte \textit{djelomično jednačivih/usporedivih} vrijednosti.
\end{itemize}

