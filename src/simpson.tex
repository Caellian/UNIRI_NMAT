\section{Simpsonova formula}

\textbf{Simpsonova formula} je nešto složeniji primjer zatvorenih Newton-Cotesovih
formula koji dobijemo kad podintegralnu funkciju $f$ interpoliramo kvadratnom
funkcijom u čvorovima interpolacije:
\begin{gather*}
    x_0 = a,\quad x_1=\dfrac{a+b}{2},\quad x_2 = b,\\
    I = \int_a^b f(x)dx \approx I^* = \int_a^b P_2(x)dx.
\end{gather*}

Geometrijska interpretacija simpsonove formule je aproksimacije površine ispod krivulje površinom ispod parabole.

Simpsonova formula je oblika: $\displaystyle
I^* = \int_a^b P_2(x)dx = \dfrac{b-a}{6}\left(f(a) + 4f\left(\dfrac{a+b}{2}\right) + f(b)\right)
$

\subsection{Ocjena greške}

Neka je $f\in C^4[a,b]$. Tada postoji $\xi \in \langle a,b\rangle$ takav da vrijedi
$$
I - I^* = \int_a^b f(x) dx - \dfrac{(b-a)}{6}\left(f(a) + 4f\left(\dfrac{a+b}{2}\right) + f(b)\right) = -(b-a)^5\dfrac{f^{(4)}(\xi)}{2880}.
$$

Ako označimo $\displaystyle M_4 = \max_{x\in[a,b]}|f^{(4)}(x)|$ onda je apsolutna
greška $\displaystyle \Delta I^* = |I - I^*| \leq \left(\dfrac{(b-a)}{2}\right)^5 \cdot \dfrac{M_4}{90}.$


\subsection{Algoritam za simpsonovu formulu}

Pythonov \codepkg{scipy.integrate} modul sadrži \codefn{simps} funkciju koja kao
argumente uzima numpy polja rezultata i ulaznih vrijednosti te integrira
koristeći simpsonovu formulu.

\begin{algorithmic}
\Function{Simpson}{$a,\;b,\;h,\;f,\;n$}
    \State $h \gets \dfrac{b - a}{n}$
    \State $r \gets \dfrac{f(a) + f(b)}{2}$

    \For{$i \in [1, n-1]$}
        \State $x \gets a + i \cdot h$
        \If{$i$ je paran}
            \State $r \gets r + 2 \cdot f(x)$
        \Else
            \State $r \gets r + 4 \cdot f(x)$
        \EndIf
    \EndFor

    \State \Return $(r + f(b)) \cdot \frac{h}{3}$
\EndFunction
\end{algorithmic}

\subsection{Produljena Simpsonova formula}

U cilju postizanja bolje aproksimacije $I^*$ integrala $I$, segment $[a,b]$
možemo podijeliti na $n$ podsegmenata uniformne duljine
$$
h=\frac{b-a}{n}\,,
$$
u čvorovima $x_i=a+ih, i \in [0,n]$.

Uz oznaku $y_i=f(x_i), i \in [0,n]$ redom, na po dva podsegmenata primjenjujemo
Simpsonovu formulu te tako dobivamo produljenu Simpsonovu formulu.

Vrijedi:

\begin{gather*}
    I = I^* + E_n\,,\\
    I^* \approx \frac{h}{3}(y_0+y_{2m}+4(y_1+\dots+y_{2m-1})+2(y_2+\dots+y_{2m-2}))\,,\\
    E_n = -\frac{b-a}{180}h^4f^{(4)}(\xi),\, \xi\in\langle a, b \rangle\,.
\end{gather*}

\pagebreak

\subsubsection{Ocjena greške}

Ukupna greška produljene Simpsonove formule je

$$
I - I^* = -h^5\sum_{i=0}^{\frac{n}{2}-1}\frac{f^{(4)}(\xi)}{90} = \frac{(b-a)h^4}{180}f^{(4)}(\xi)\,.
$$

Iz ocjene greške moguće je kontrolirati grešku aproksimacije. Za zadanu točnost
aproksimacije $\varepsilon$ možemo unaprijed odrediti potrebam broj podsegmenata

$$
n\geq \sqrt[4]{\frac{(b-a)^5M_4}{180\cdot \varepsilon}}\,,\quad M_4 = \max_{x\in[a,b]}|f^{(4)}(x)|.
$$
