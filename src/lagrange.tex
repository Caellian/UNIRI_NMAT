\section{Lagrangeov interpolacijski polinom}

Formula za Lagrangeov interpolacijski polinom (${\mathbf L}_n$) je:

\begin{align}
    {\mathbf I}_i(x) &\coloneq \prod_{\substack{j=0\\j\neq i}}^{n} \frac{x-x_j}{x_i-x_j}\\
    {\mathbf L}_n(x) &\coloneq \sum_{i=0}^{n} y_i\mathbf{I}_i(x)
\end{align}

Lagrangeov interpolacijski polinom je bolji od korištenja vandermondeove determinante jer su sustavi dobiveni vandermondeovom determinantom nestabilni.
Također je i postupak brži jer ne zahtijeva računanje inverzne matrice koja u računalnoj primjeni zahtijeva mnogo međukoraka i instrukcija.

\begin{example}[uporaba lagrangeovog polinoma]
    Odredi interpolacijski polinom koristeći lagrangeove polinome ako su zadane sljedeće točke:

    \center
    \begin{tabular}{r|c|c|c}
        $x$&-3&4&3\\
        \hline
        $y$&-4&2&0\\
    \end{tabular}
\end{example}

Računamo $\mathrm{I}_0(x), \mathrm{I}_1(x),$ i $\mathrm{I}_2(x)$ za točke:

$$
\mathrm{I}_0(x) = \frac{x-x_1}{x_0-x_1} \cdot \frac{x-x_2}{x_0-x_2} = \frac{x-4}{-3-4} \cdot \frac{x-3}{-3-3} = \frac{(x-4)(x-3)}{-7\cdot-6} = \frac{x^2-7x+12}{42}
$$

\begin{warningbox}[nedostajući član]
    $$
    \frac{x-x_j}{x_0-x_j}=\frac{x-x_0}{x_0-x_0}=\frac{x-x_0}{0}
    $$
\end{warningbox}

\begin{align*}
\mathrm{I}_1(x) &= \frac{(x-x_0)(x-x_2)}{(x_1-x_0)(x_1-x_2)} = \frac{(x+3)(x-3)}{7\cdot1} = \frac{x^2-9}{7}\\\\
\mathrm{I}_2(x) &= \frac{(x-x_0)(x-x_1)}{(x_2-x_0)(x_2-x_1)} = \frac{(x+3)(x-4)}{6\cdot-1} = \frac{x^2-x-12}{-6}
\end{align*}

\begin{multicols}{2}

\vspace*{0pt}

Nakon uvrštavanja dobivamo polinom:
\begin{align*}
\mathrm{L}_2(x) &= y_0\mathrm{I}_0(x) + y_1\mathrm{I}_1(x) + y_2\mathrm{I}_2(x)\\
&= -4\cdot\frac{x^2-7x+12}{42} + 2\cdot\frac{x^2-9}{7} + 0\cdot\frac{x^2-x-12}{-6}\\
&= \frac{-2x^2+14x-24+6x^2-54}{21}\\
&= \frac{4x^2}{21} + \frac{2x\cdot\cancel{7}}{3\cdot\cancel{7}} - \frac{26\cdot\cancel{3}}{7\cdot\cancel{3}}\\
&\boxed{= {\frac{4}{21}}x^2 + {\frac{2}{3}}x - {\frac{26}{7}}}
\end{align*}

\vspace*{0pt}

\newcolumn

\vspace*{0pt}

\begin{tikzpicture}
    \begin{axis}[
        xmin=-4,xmax=4.5,
        ymin=-5,ymax=3
    ]
        \addplot[plotlinefound] expression[samples=100]{4/21*x^2 + 2/3*x - 26/7} node[pos=0.5, anchor=north west]{$y= \sin{2x} + \frac{1}{2}$};
        \addplot[plotinterpolated] coordinates {(-3,-4)};
        \addplot[plotinterpolated] coordinates {(4,2)};
        \addplot[plotinterpolated] coordinates {(3,0)};
    \end{axis}
\end{tikzpicture}

\end{multicols}

\subsection{Uporaba Lagrangeovog polinoma u Pythonu}

\begin{example}
    Izračunajte vrijednost Lagrangeovog polinoma iz prethodnog primjera za točku
    $x=3.5$ u Pythonu.
\end{example}

\section{Lagrangeov interpolacijski polinom}

Formula za Lagrangeov interpolacijski polinom (${\mathbf L}_n$) je:

\begin{align}
    {\mathbf I}_i(x) &\coloneq \prod_{\substack{j=0\\j\neq i}}^{n} \frac{x-x_j}{x_i-x_j}\\
    {\mathbf L}_n(x) &\coloneq \sum_{i=0}^{n} y_i\mathbf{I}_i(x)
\end{align}

Lagrangeov interpolacijski polinom je bolji od korištenja vandermondeove determinante jer su sustavi dobiveni vandermondeovom determinantom nestabilni.
Također je i postupak brži jer ne zahtijeva računanje inverzne matrice koja u računalnoj primjeni zahtijeva mnogo međukoraka i instrukcija.

\begin{example}[uporaba lagrangeovog polinoma]
    Odredi interpolacijski polinom koristeći lagrangeove polinome ako su zadane sljedeće točke:

    \center
    \begin{tabular}{r|c|c|c}
        $x$&-3&4&3\\
        \hline
        $y$&-4&2&0\\
    \end{tabular}
\end{example}

Računamo $\mathrm{I}_0(x), \mathrm{I}_1(x),$ i $\mathrm{I}_2(x)$ za točke:

$$
\mathrm{I}_0(x) = \frac{x-x_1}{x_0-x_1} \cdot \frac{x-x_2}{x_0-x_2} = \frac{x-4}{-3-4} \cdot \frac{x-3}{-3-3} = \frac{(x-4)(x-3)}{-7\cdot-6} = \frac{x^2-7x+12}{42}
$$

\begin{warningbox}[nedostajući član]
    $$
    \frac{x-x_j}{x_0-x_j}=\frac{x-x_0}{x_0-x_0}=\frac{x-x_0}{0}
    $$
\end{warningbox}

\begin{align*}
\mathrm{I}_1(x) &= \frac{(x-x_0)(x-x_2)}{(x_1-x_0)(x_1-x_2)} = \frac{(x+3)(x-3)}{7\cdot1} = \frac{x^2-9}{7}\\\\
\mathrm{I}_2(x) &= \frac{(x-x_0)(x-x_1)}{(x_2-x_0)(x_2-x_1)} = \frac{(x+3)(x-4)}{6\cdot-1} = \frac{x^2-x-12}{-6}
\end{align*}

\begin{multicols}{2}

\vspace*{0pt}

Nakon uvrštavanja dobivamo polinom:
\begin{align*}
\mathrm{L}_2(x) &= y_0\mathrm{I}_0(x) + y_1\mathrm{I}_1(x) + y_2\mathrm{I}_2(x)\\
&= -4\cdot\frac{x^2-7x+12}{42} + 2\cdot\frac{x^2-9}{7} + 0\cdot\frac{x^2-x-12}{-6}\\
&= \frac{-2x^2+14x-24+6x^2-54}{21}\\
&= \frac{4x^2}{21} + \frac{2x\cdot\cancel{7}}{3\cdot\cancel{7}} - \frac{26\cdot\cancel{3}}{7\cdot\cancel{3}}\\
&\boxed{= {\frac{4}{21}}x^2 + {\frac{2}{3}}x - {\frac{26}{7}}}
\end{align*}

\vspace*{0pt}

\newcolumn

\vspace*{0pt}

\begin{tikzpicture}
    \begin{axis}[
        xmin=-4,xmax=4.5,
        ymin=-5,ymax=3,
        ]
        \addplot[plotlinefound] expression[samples=100]{4/21*x^2 + 2/3*x - 26/7}
          node[pos=0.5, anchor=north west] {$y=\sin(2x) + \frac{1}{2}$};
        \addplot[plotinterpolated] coordinates {(-3,-4)};
        \addplot[plotinterpolated] coordinates {(4,2)};
        \addplot[plotinterpolated] coordinates {(3,0)};
    \end{axis}
\end{tikzpicture}

\end{multicols}

\subsection{Nedostaci}

\begin{definition}
    Funkcija $g(x)$ je reda $\mathcal{O}(h(x))$ za $x \to x_0$, odnosno
    $$
    g(x) = \mathcal{O}(h(x)), x\to x_0
    $$
    ako postoje konstante $\sigma, C$ takve da
    $$
    |x-x_0| \leq \sigma \Rightarrow |g(x)| \leq C|h(x)|
    $$
\end{definition}

\subsubsection{Veliki broj aritmetičkih operacija}

Interpolacija lagrangeovim polinomom zahtijeva $\mathcal{O}(n^2)$ operacija kod prvog izvrednjavanja i $\mathcal{O}(n)$ operacija za svako sljedeće izvrednjavanje.

Kod računalne primjene brzina izračuna ovisi u broju primjena dobivenog polinoma jer se unaprijed izračunata vandermondeova determinanta za sustav može primijeniti više puta dok je u slučaju lagrangeovih polinoma potrebno koristiti pokazivače na članove polinoma ($\mathrm{I}_n$, koji su parcijalno ispunjene funkcije na gomili) što npr. za crtanje funkcije \textbf{može zahtijevati veći broj dereferenciranja s gomile}.

Taj problem je rješiv korištenjem biblioteka za simboličku matematiku, no to povećava veličinu distribuirane binarne datoteke, iako je taj kompromis generalno primjeren ako se biblioteka koristi za pojednostavljivanje više izraza tokom izvođenja programa ili se radi o jednonamjenskom programu.

\subsubsection{Velika zavisnost o ulaznim podacima}

U slučaju izmjena ulaznih podataka (ponajviše x koordinata) je potrebno ponovo računati sve polinome interpolacije ($\mathrm{I}_n$).

Kod računalne primjene, prethodno navedene parcijalno ispunjene funkcije (za $\mathrm{I}_n$) je potrebno ponovo slagati - \textbf{loša memoizacija}.

\subsubsection{Slaba aproksimacija i oscilacije}

Ako se odabere polinom preniskog stupnja, on može \textbf{slabo aproksimirati originalnu funkciju}.

\textbf{Rungeov fenomen}: Ako se odabere polinom previsokog stupnja, na dijelovima se mogu pojaviti \textbf{prevelike oscilacije}, tj. greške na rubovima intervala aproksimacije.


\newpage

\subsection{Nedostaci}

\begin{definition}
    Funkcija $g(x)$ je reda $\mathcal{O}(h(x))$ za $x \to x_0$, odnosno
    $$
    g(x) = \mathcal{O}(h(x)), x\to x_0
    $$
    ako postoje konstante $\sigma, C$ takve da
    $$
    |x-x_0| \leq \sigma \Rightarrow |g(x)| \leq C|h(x)|
    $$
\end{definition}

\subsubsection{Veliki broj aritmetičkih operacija}

Interpolacija lagrangeovim polinomom zahtijeva $\mathcal{O}(n^2)$ operacija kod prvog izvrednjavanja i $\mathcal{O}(n)$ operacija za svako sljedeće izvrednjavanje.

Kod računalne primjene brzina izračuna ovisi u broju primjena dobivenog polinoma jer se unaprijed izračunata vandermondeova determinanta za sustav može primijeniti više puta dok je u slučaju lagrangeovih polinoma potrebno koristiti pokazivače na članove polinoma ($\mathrm{I}_n$, koji su parcijalno ispunjene funkcije na gomili) što npr. za crtanje funkcije \textbf{može zahtijevati veći broj dereferenciranja s gomile}.

Taj problem je rješiv korištenjem biblioteka za simboličku matematiku, no to povećava veličinu distribuirane binarne datoteke, iako je taj kompromis generalno primjeren ako se biblioteka koristi za pojednostavljivanje više izraza tokom izvođenja programa ili se radi o jednonamjenskom programu.

\subsubsection{Velika zavisnost o ulaznim podacima}

U slučaju izmjena ulaznih podataka (ponajviše $x$ koordinata) je potrebno ponovo računati sve polinome interpolacije ($\mathrm{I}_n$).

Kod računalne primjene, prethodno navedene parcijalno ispunjene funkcije (za $\mathrm{I}_n$) je potrebno ponovo slagati - \textbf{loša memoizacija}.

\subsubsection{Slaba aproksimacija i oscilacije}

Ako se odabere polinom preniskog stupnja, on može \textbf{slabo aproksimirati originalnu funkciju}.

\textbf{Rungeov fenomen}: Ako se odabere polinom previsokog stupnja, na dijelovima se mogu pojaviti \textbf{prevelike oscilacije}, tj. greške na rubovima intervala aproksimacije.
