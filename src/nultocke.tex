Cilj numeričkih metoda za rješavanje nelinearnih jednadžbi je odrediti
\textbf{nultočku (korijen, rješenje)} neprekidne ralne funkcije $f$ na segmentu
$[a,b]$, odnosno rješiti
$$
    f(x) = 0\,.
$$

Moguće je da
\begin{itemize}
    \item funkcija $f$ ima više realnih nultočaka i da su neke višestruke, kao i
    da
    \item funkcija $f$ ima kompleksne nultočke (tj. nema realne nultočke),
\end{itemize}

Ako za funkciju $f$ vrijedi $f(a)f(b) \leq 0$, onda ona na segmentu $[a,b]$
mijenja predznak te postoji \textit{barem jedna} nultočka. Ako dodatno znamo da
je funkcija $f$ monotona na tom segmentu, onda znamo i da je nultočka
jedinstvena.

Traženje nultočki do na zadanu točnost sastoji se od dvije faze:
\begin{enumerate}
    \item Određivanje što manjeg segmenta u kojem se nalazi nultočka funkcije
    \item Nalaženje nultočke iterativnom metodom do na traženu točnost
\end{enumerate}

Osnovni korak numeričkih metoda je izolirati jedan korijen (nultočku, rješenje)
$\xi$ jednadžbe $f(x) = 0$. Nakon što se izolira jedan korijen jednadžbe nekom
odabranom numeričkom metodom formiramo niz aproksimacija $x_i \to \xi$.
Numeričku metodu zaustavljamo kad se zadovolji zadani kriterij točnosti
$$
    |\xi - x_i | < \varepsilon\,.
$$

Vrlo često nije jednostavno prepoznati kad smo dovoljno blizu korijena, ali iz
Teorema o srednjoj vrijednosti slijedi
$$
    |\xi - x_i| = \left| \frac{f(x_i)}{f'(c)} \right|, c \in [\xi, x_i]\,.
$$

\begin{definition}[red konvergencije]
    Neka niz $(x_i)_{i\in\mathbb{N}}$ dobiven nekom iterativnom metodom
    konvergira prema $\xi$. Ako postoje dvije pozitivne konstante $C$, $r$,
    takve da vrijedi
    $$
        \lim_{i\to\infty} \left|\xi-x_{i+1}\right| \leq C|\xi-x_i|^r\,,
    $$
    tada kažemo da \textbf{metoda ima red konvergencije} $r$. 
\end{definition}
