\section{Arclen aproksimacija bezierove krivulje}

U računalnoj grafici je česta uporaba bezierovih (i drugih parametarski zadanih)
krivulja. Linearna interpolacija po parametarski zadanim krivuljama, određivanje
duljine krivulje, određivanje graničnog okvira (kvadrat najmanje površine koji
ne presijeca krivulju), i mnoge druge primjene su komplicirane jer takve
krivulje nije moguće razdjeliti na djelove jednake duljine algebarskim putem.

Kako bi takve krivulje mogli razdjeliti, potrebno je koristiti numeričke metode
(metodu regresije).

\bigskip
Bezierove krivulje zadane parametarski sa $t$:

\begin{equation*}
    P(t) = (1-t)^3\cdot P_1 + 3 (1-t)^2 t\cdot P_2 +3(1-t) t^2 \cdot P_3 + t^3 \cdot P_4\,,
\end{equation*}
