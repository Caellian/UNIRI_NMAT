\section{Crtanje grafova}

Za crtanje grafova se u Pythonu najčešće koristi \codepkg{matplotlib}
biblioteka. Ista biblioteka je dostupna i kroz \codepkg{pylab} biblioteku:

\input{../code/grafovi_import}

Crtanje se provodi pomoću \codefn{plot} funkcije.

\section{Crtanje grafova}

Za crtanje grafova se u Pythonu najčešće koristi \codepkg{matplotlib}
biblioteka. Ista biblioteka je dostupna i kroz \codepkg{pylab} biblioteku:

\input{../code/grafovi_import}

Crtanje se provodi pomoću \codefn{plot} funkcije.

\section{Crtanje grafova}

Za crtanje grafova se u Pythonu najčešće koristi \codepkg{matplotlib}
biblioteka. Ista biblioteka je dostupna i kroz \codepkg{pylab} biblioteku:

\input{../code/grafovi_import}

Crtanje se provodi pomoću \codefn{plot} funkcije.

\section{Crtanje grafova}

Za crtanje grafova se u Pythonu najčešće koristi \codepkg{matplotlib}
biblioteka. Ista biblioteka je dostupna i kroz \codepkg{pylab} biblioteku:

\input{../code/grafovi_import}

Crtanje se provodi pomoću \codefn{plot} funkcije.

\input{../code/grafovi}







\subsection{Format teksta za stil}

Format je oblika \verb|fmt = '[marker][line][color]'|, drukčiji redoslijed
znakova je isto podržan ali može dati neočekivane rezultate.

\begin{itemize}
    \item Popis \verb|marker|a je dostupan u \href{https://matplotlib.org/stable/api}{službenoj dokumentaciji} pod \href{https://matplotlib.org/stable/api/markers_api.html}{\codepkg{matplotlib.markers}}.
    \item \verb|line| može biti:
    \begin{itemize}
        \item '-' za punu liniju
        \item '--' za iscrtkanu liniju
        \item '-.' za crta-točka-crta liniju
        \item ':' za istočkanu liniju
    \end{itemize}
    \item \verb|color| podržava kratke nazive:
    \begin{itemize}
        \item 'b' za plavu boju
        \item 'g' za zelenu boju
        \item 'r' za crvenu boju
        \item 'c' za tirkiznu boju
        \item 'm' za magenta boju
        \item 'y' za žutu boju
        \item 'k' za crnu boju
        \item 'w' za bijelu boju
    \end{itemize}
\end{itemize}

\subsection{Parametri za stil}

Osim teksta za format je moguće koristiti i imenovane paremetre poput
\codenamed{alpha}, \codenamed{color}, \codenamed{dashes}, \codenamed{fillstyle},
\codenamed{linestyle}, \codenamed{gapcolor}, \codenamed{marker},
\codenamed{markeredgecolor}, \codenamed{markeredgewidth},
\codenamed{markersize}, \codenamed{markevery}, itd.
