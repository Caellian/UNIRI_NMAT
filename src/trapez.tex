\subsection{Trapezna formula}

Najjednostavniji primjer zatvorenih Newton-Cotesovih formula dobijemo kad
podintegralnu funkciju $f$ interpoliramo linearnom funkcijom u čvorovima
interpolacije.

Geometrijska interpretacija trapezne formule je aproksimacija površine ispod
krivulje površinom trapeza.

Trapezna formula ima oblik: $\displaystyle
I^* = \int_a^b P_1(x)dx = (b-a)\dfrac{f(a)+f(b)}{2}
$

\subsubsection{Ocjena greške}

Neka je $f\in C^2[a,b]$. Tada postoji $\xi \in [a,b]$ takav da vrijedi
$$
I - I^* = \int_a^b f(x) dx - (b-a)\dfrac{f(a)+f(b)}{2} = -(b-a)^3\dfrac{f''(\xi)}{12}.
$$

Ako označimo $\displaystyle M_2 = \max_{x\in[a,b]}|f''(x)|$ onda je apsolutna
greška $\displaystyle \Delta I^* = |I - I^*| \leq \dfrac{(b-a)^3}{12}M_2.$

\subsubsection{Algoritam}

Pythonov \verb|scipy.integrate| modul sadrži \verb|trapz| funkciju koja kao
argumente uzima numpy polja rezultata i ulaznih vrijednosti te integrira
koristeći trapeznu formulu.

Funkcija za integraciju na intervalu $[a, b]$ uz $n$ trapeza je:
\begin{algorithmic}
\Function{Trapez}{$a,\;b,\;h,\;f,\;n$}
    \State $h \gets \dfrac{b - a}{n}$
    \State $r \gets \frac{f(a) + f(b)}{2}$
    
    \For{$i \in [1, n-1]$}
      \State $x \gets a + i \cdot h$
      \State $r \gets r + f(x)$
    \EndFor
    
    \State \Return $r \cdot h$
\EndFunction
\end{algorithmic}

\subsubsection{Produljena trapezna formula}

Početni segment $[a,b]$ dijelimo na $n$ podsegmenata uniformne duljine
$$
h = \frac{b-a}{n}\,.
$$

Integral aproksimiramo sumom rezultata trapezne formule po dijelovima
$$
I = \int_{b}^{a} f(x)dx = \sum_{i=0}^{n-1} \int_{x_i}^{x_{i+1}}f(x)dx \approx h \sum_{i=0}^{n-1} \frac{f(x_i)+f(x_{i+1})}{2}\,.
$$

Produljena trapezna formula se najčešće zapisuje u obliku
$$
I^* = \int_{a}^{b} f(x)dx \approx \frac{h}{2}(y_0+2y_1+\dots+1y_{n-1}+y_n)\,,
$$
gdje je $f(x_i) = y_i, i\in [0, n]$.


%$\sin{1.3}^{x}-\frac{x}{8}+4$
% \begin{tikzpicture}
%     \begin{axis}[
%         xmin=-1,xmax=5,
%         ymin=-1,ymax=6,
%     ]
%         \addplot[plotlinefound] expression[samples=100]{sin(1.3)^\x-\x/8+4} node[pos=0.8,anchor=south west]{$f(x)$};
%         \addplot[plotinterpolated] coordinates {(0,3)};
%         \addplot[plotinterpolated] coordinates {(2,4)};
%         \addplot[plotinterpolated] coordinates {(4,2)};
%     \end{axis}
% \end{tikzpicture}

\bigskip
\noindent
\textbf{Ocjena greške}

Ukupna greška produljene trašezne formule je
$$
I - I^* = -h^3 \sum_{i=0}^{n-1} \frac{f''(\xi_i)}{12} = \frac{(b-a)h^2}{12}f''(\xi)\,.
$$

Iz ocjene greške je moguće kontrolirati grešku aproksimacije.
Za zadanu točnost aproksimacije $\varepsilon$ u nekim slučajevima možemo i unaprijed odrediti potreban broj podsegmenata
$$
n \geq \sqrt{\frac{(b-a)^3M_2}{12 \cdot \varepsilon}}\,, M_2 = \max_{x\in[a,b]}|f''(x)|\,.
$$
