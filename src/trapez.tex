\section{Numeričko integriranje}

Postoji veliki broj funkcija kod kojih nije moguće ili je izrazito teško
analitički odrediti vrijednost integrala. U tim slučajevima se mogu koristiti
numeričke metode za \textbf{približno} rješavanje.

Aproksimacija integrala numeričkim metodama integracije se provodi određivanjem
interpolacijskog polinoma $n$-tog stupnja:
$$
P_n(x) \approx f(x).
$$

Ako je polinom $P_n(x)$ dobra aproksimacija podintegralne funkcije, tada vrijedi
$$
I = \int_a^b f(x) dx \approx I^* = \int_a^b P_n(x) dx.
$$

Interpolaciju funkcije $f$ je moguće odrediti na više načina, ali konačne
integracijske formule imaju oblik
$$
I^* = \sum_{i=0}^n w_i f(x_i),
$$
gdje je $n$ unaprijed zadan, $x_i$ su čvorovi integracije (interpolacije), a
$w_i$ su težine.

Ako su čvorovi integracijskih formula fiksni, onda takve formule nazivamo
\textbf{Newton-Cotesove formule}. Ako se čvorovi bitaju tako da se dobije veća
točnost formule, tada takve formule zovemo \textbf{Gaussove formule}.

U praksi se koriste dva tipa Newton-Cotesovih formula:
\begin{itemize}
    \item zatvorene formule - rubovi segmenata $a$ i $b$ su ujedno i čvorovi,
    \item otvorene formule - rubovi segmenata $a$ i $b$ nisu čvorovi.
\end{itemize}

\subsection{Trapezna formula}

Najjednostavniji primjer zatborenih Newton-Cotesovih formula dobijemo kad
podintegralnu funkciju $f$ interpoliramo linearnom funkcijom u čvorovima
interpolacije.

Geometrijska interpretacija trapezne formule je aproksimacija površine ispod
krivulje površinom trapeza.

Trapezna formula ima oblik: $\displaystyle
I^* = \int_a^b P_1(x)dx = (b-a)\dfrac{f(a)+f(b)}{2}
$

\subsubsection{Ocjena greške}

Neka je $f\in C^2[a,b]$. Tada postoji $\xi \in [a,b]$ takav da vrijedi
$$
I - I^* = \int_a^b f(x) dx - (b-a)\dfrac{f(a)+f(b)}{2} = -(b-a)^3\dfrac{f''(\xi)}{12}.
$$

Ako označimo $\displaystyle M_2 = \max_{x\in[a,b]}|f''(x)|$ onda je apsolutna
greška $\displaystyle \Delta I^* = |I - I^*| \leq \dfrac{(b-a)^3}{12}M_2.$

\subsubsection{Algoritam}

Pythonov \verb|scipy.integrate| modul sadrži \verb|trapz| funkciju koja kao
argumente uzima numpy polja rezultata i ulaznih vrijednosti te integrira
koristeći trapeznu formulu.

Funkcija za integraciju na intervalu $[a, b]$ uz $n$ trapeza je:
\begin{algorithmic}
\Function{Trapez}{$a,\;b,\;h,\;f,\;n$}
    \State $h \gets \dfrac{b - a}{n}$
    \State $r \gets \frac{f(a) + f(b)}{2}$
    
    \For{$i \in [1, n-1]$}
      \State $x \gets a + i \cdot h$
      \State $r \gets r + f(x)$
    \EndFor
    
    \State \Return $r \cdot h$
\EndFunction
\end{algorithmic}
