\section{Metoda sekante}

Metoda sekante premošćuje prethodno navedene probleme metode tangente.

Aproksimiramo li derivaciju $f'(x_n)$ izrazom
$$
\dfrac{f(x_n)-f(x_{n-1})}{x_n - x_{n-1}},
$$
iz \fancyeq[izraza]{eq:metoda_tangente}, dobivamo
$$
x_{n+1} = x_n - f(x_n)\dfrac{x_n-x_{n-1}}{f(x_n)-f(x_{n-1})}.
$$

Kako je vrijednost $f(x_{n-1})$ poznata iz prethodne iteracije, metoda sekante u
svakom koraku treba samo jedno računanje funkcije $f$. Zbog toga je metoda
sekante duplo \textit{jeftinija} od Newtonove metode koja u svakom koraku
zahtijeva računanje vrijednosti $f$ i $f'$.

Uz zadanu točnost $\varepsilon$ kriterij zaustavljanja je:

$$
|x_{n+1}-x_n| = |f(x_n)|\cdot\left|\frac{x_n-x_{n-1}}{f(x_n)-f(x_{n-1})}\right|
$$

\subsection{Brzina konvergencije}

Ako su početne aproksimacije $x_0$ i $x_1$ birane dovoljno blizu rješenja, tada
uz uvjete $f'(x)\neq 0$ i $f''(x)\neq 0$, metoda sekante konvergira brzinom
$\frac{1+\sqrt{5}}{2}\approx 1.618$.

\begin{multicols}{2}

\subsection{Algoritam za metodu sekante}

\begin{algorithmic}
\Function{Sekanta}{$a,\;b,\;\varepsilon,\;f,\;n$}
    \State $i \gets 0$
    \While{$\lvert b - a \rvert > \varepsilon$}
        \State $i \gets i + 1$
        \State $x_i \gets b - \dfrac{f(b) \cdot (b - a)}{f(b) - f(a)}$
        \If{$f(x_i) = 0$}
            \State \Return $x_i,\;i$ \Comment{Egzaktno rješenje}
        \Else
            \State $a \gets b$
            \State $b \gets x_i$
        \EndIf
        \If{$i > n$}
            \Error{Rješenje nije nađeno u $n$ iteracija.}
        \EndIf
    \EndWhile
    \State \Return $b,\,i$
\EndFunction
\end{algorithmic}

\columnbreak

\subsection{Metoda sekante u Pythonu}

\begingroup
\makeatletter
\@totalleftmargin=-1cm
\section{Metoda sekante}

Metoda sekante premošćuje prethodno navedene probleme metode tangente.

Aproksimiramo li derivaciju $f'(x_n)$ izrazom
$$
\dfrac{f(x_n)-f(x_{n-1})}{x_n - x_{n-1}},
$$
iz \fancyeq[izraza]{eq:metoda_tangente}, dobivamo
$$
x_{n+1} = x_n - f(x_n)\dfrac{x_n-x_{n-1}}{f(x_n)-f(x_{n-1})}.
$$

Kako je vrijednost $f(x_{n-1})$ poznata iz prethodne iteracije, metoda sekante u
svakom koraku treba samo jedno računanje funkcije $f$. Zbog toga je metoda
sekante duplo \textit{jeftinija} od Newtonove metode koja u svakom koraku
zahtijeva računanje vrijednosti $f$ i $f'$.

Uz zadanu točnost $\varepsilon$ kriterij zaustavljanja je:

$$
|x_{n+1}-x_n| = |f(x_n)|\cdot\left|\frac{x_n-x_{n-1}}{f(x_n)-f(x_{n-1})}\right|
$$

\subsection{Brzina konvergencije}

Ako su početne aproksimacije $x_0$ i $x_1$ birane dovoljno blizu rješenja, tada
uz uvjete $f'(x)\neq 0$ i $f''(x)\neq 0$, metoda sekante konvergira brzinom
$\frac{1+\sqrt{5}}{2}\approx 1.618$.

\begin{multicols}{2}

\subsection{Algoritam za metodu sekante}

\begin{algorithmic}
\Function{Sekanta}{$a,\;b,\;\varepsilon,\;f,\;n$}
    \State $i \gets 0$
    \While{$\lvert b - a \rvert > \varepsilon$}
        \State $i \gets i + 1$
        \State $x_i \gets b - \dfrac{f(b) \cdot (b - a)}{f(b) - f(a)}$
        \If{$f(x_i) = 0$}
            \State \Return $x_i,\;i$ \Comment{Egzaktno rješenje}
        \Else
            \State $a \gets b$
            \State $b \gets x_i$
        \EndIf
        \If{$i > n$}
            \Error{Rješenje nije nađeno u $n$ iteracija.}
        \EndIf
    \EndWhile
    \State \Return $b,\,i$
\EndFunction
\end{algorithmic}

\columnbreak

\subsection{Metoda sekante u Pythonu}

\begingroup
\makeatletter
\@totalleftmargin=-1cm
\section{Metoda sekante}

Metoda sekante premošćuje prethodno navedene probleme metode tangente.

Aproksimiramo li derivaciju $f'(x_n)$ izrazom
$$
\dfrac{f(x_n)-f(x_{n-1})}{x_n - x_{n-1}},
$$
iz \fancyeq[izraza]{eq:metoda_tangente}, dobivamo
$$
x_{n+1} = x_n - f(x_n)\dfrac{x_n-x_{n-1}}{f(x_n)-f(x_{n-1})}.
$$

Kako je vrijednost $f(x_{n-1})$ poznata iz prethodne iteracije, metoda sekante u
svakom koraku treba samo jedno računanje funkcije $f$. Zbog toga je metoda
sekante duplo \textit{jeftinija} od Newtonove metode koja u svakom koraku
zahtijeva računanje vrijednosti $f$ i $f'$.

Uz zadanu točnost $\varepsilon$ kriterij zaustavljanja je:

$$
|x_{n+1}-x_n| = |f(x_n)|\cdot\left|\frac{x_n-x_{n-1}}{f(x_n)-f(x_{n-1})}\right|
$$

\subsection{Brzina konvergencije}

Ako su početne aproksimacije $x_0$ i $x_1$ birane dovoljno blizu rješenja, tada
uz uvjete $f'(x)\neq 0$ i $f''(x)\neq 0$, metoda sekante konvergira brzinom
$\frac{1+\sqrt{5}}{2}\approx 1.618$.

\begin{multicols}{2}

\subsection{Algoritam za metodu sekante}

\begin{algorithmic}
\Function{Sekanta}{$a,\;b,\;\varepsilon,\;f,\;n$}
    \State $i \gets 0$
    \While{$\lvert b - a \rvert > \varepsilon$}
        \State $i \gets i + 1$
        \State $x_i \gets b - \dfrac{f(b) \cdot (b - a)}{f(b) - f(a)}$
        \If{$f(x_i) = 0$}
            \State \Return $x_i,\;i$ \Comment{Egzaktno rješenje}
        \Else
            \State $a \gets b$
            \State $b \gets x_i$
        \EndIf
        \If{$i > n$}
            \Error{Rješenje nije nađeno u $n$ iteracija.}
        \EndIf
    \EndWhile
    \State \Return $b,\,i$
\EndFunction
\end{algorithmic}

\columnbreak

\subsection{Metoda sekante u Pythonu}

\begingroup
\makeatletter
\@totalleftmargin=-1cm
\section{Metoda sekante}

Metoda sekante premošćuje prethodno navedene probleme metode tangente.

Aproksimiramo li derivaciju $f'(x_n)$ izrazom
$$
\dfrac{f(x_n)-f(x_{n-1})}{x_n - x_{n-1}},
$$
iz \fancyeq[izraza]{eq:metoda_tangente}, dobivamo
$$
x_{n+1} = x_n - f(x_n)\dfrac{x_n-x_{n-1}}{f(x_n)-f(x_{n-1})}.
$$

Kako je vrijednost $f(x_{n-1})$ poznata iz prethodne iteracije, metoda sekante u
svakom koraku treba samo jedno računanje funkcije $f$. Zbog toga je metoda
sekante duplo \textit{jeftinija} od Newtonove metode koja u svakom koraku
zahtijeva računanje vrijednosti $f$ i $f'$.

Uz zadanu točnost $\varepsilon$ kriterij zaustavljanja je:

$$
|x_{n+1}-x_n| = |f(x_n)|\cdot\left|\frac{x_n-x_{n-1}}{f(x_n)-f(x_{n-1})}\right|
$$

\subsection{Brzina konvergencije}

Ako su početne aproksimacije $x_0$ i $x_1$ birane dovoljno blizu rješenja, tada
uz uvjete $f'(x)\neq 0$ i $f''(x)\neq 0$, metoda sekante konvergira brzinom
$\frac{1+\sqrt{5}}{2}\approx 1.618$.

\begin{multicols}{2}

\subsection{Algoritam za metodu sekante}

\begin{algorithmic}
\Function{Sekanta}{$a,\;b,\;\varepsilon,\;f,\;n$}
    \State $i \gets 0$
    \While{$\lvert b - a \rvert > \varepsilon$}
        \State $i \gets i + 1$
        \State $x_i \gets b - \dfrac{f(b) \cdot (b - a)}{f(b) - f(a)}$
        \If{$f(x_i) = 0$}
            \State \Return $x_i,\;i$ \Comment{Egzaktno rješenje}
        \Else
            \State $a \gets b$
            \State $b \gets x_i$
        \EndIf
        \If{$i > n$}
            \Error{Rješenje nije nađeno u $n$ iteracija.}
        \EndIf
    \EndWhile
    \State \Return $b,\,i$
\EndFunction
\end{algorithmic}

\columnbreak

\subsection{Metoda sekante u Pythonu}

\begingroup
\makeatletter
\@totalleftmargin=-1cm
\input{../code/sekanta}
\endgroup

\end{multicols}

\endgroup

\end{multicols}

\endgroup

\end{multicols}

\endgroup

\end{multicols}
