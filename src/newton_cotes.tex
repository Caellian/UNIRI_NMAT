\section{Numeričko integriranje}

Postoji veliki broj funkcija kod kojih nije moguće ili je izrazito teško
analitički odrediti vrijednost integrala. U tim slučajevima se mogu koristiti
numeričke metode za \textbf{približno} rješavanje.

Aproksimacija integrala numeričkim metodama integracije se provodi određivanjem
interpolacijskog polinoma $n$-tog stupnja:
$$
P_n(x) \approx f(x).
$$

Ako je polinom $P_n(x)$ dobra aproksimacija podintegralne funkcije, tada vrijedi
$$
I = \int_a^b f(x) dx \approx I^* = \int_a^b P_n(x) dx.
$$

Interpolaciju funkcije $f$ je moguće odrediti na više načina, ali konačne
integracijske formule imaju oblik
$$
I^* = \sum_{i=0}^n f(x_i) w_i,
$$
gdje je $n$ unaprijed zadan, $x_i$ su čvorovi integracije (interpolacije), a
$w_i$ su težine.

\bigskip

\textbf{Newton-Cotesove formule} - integracijske formule sa \textit{fiksnim usaljenostima čvorovima}.

\textbf{Gaussove formule} - integracijske formule sa čvorovima odabranim s ciljem smanjenja pogreške.

\bigskip
\noindent
U praksi se koriste dva tipa Newton-Cotesovih formula:
\begin{itemize}
    \item zatvorene formule - rubovi segmenata $a$ i $b$ su ujedno i čvorovi,
    \item otvorene formule - rubovi segmenata $a$ i $b$ nisu čvorovi.
\end{itemize}

\bigskip
\noindent
Jer su težine $w_i$ izračunate iz Lagrangeovog polinoma, newton-cotesove formule
također pate od \textbf{Rungeovog fenomena}, tj. imaju velike pogreške ukoliko
je veličina koraka velika. Kako bi se mitigirala pogreška se često interval
integracije $[a,b]$ dijeli na manje podintervale te se formule primjenjuju
pojedinačno na svaki podinterval a rezultati zbrajaju. Time dobivamo
\textbf{produljene formule} (engl. \textit{composite rule}).
