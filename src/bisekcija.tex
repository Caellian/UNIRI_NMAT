\section{Određivanje nultočka}

Cilj numeričkih metoda za rješavanje nelinearnih jednadžbi je odrediti
\textbf{nultočku (korijen, rješenje)} neprekidne ralne funkcije $f$ na segmentu
$[a,b]$, odnosno rješiti
$$
    f(x) = 0\,.
$$

Moguće je da
\begin{itemize}
    \item funkcija $f$ ima više realnih nultočaka i da su neke višestruke, kao i
    da
    \item funkcija $f$ ima kompleksne nultočke (tj. nema realne nultočke),
\end{itemize}

Ako za funkciju $f$ vrijedi $f(a)f(b) \leq 0$, onda ona na segmentu $[a,b]$
mijenja predznak te postoji \textit{barem jedna} nultočka. Ako dodatno znamo da
je funkcija $f$ monotona na tom segmentu, onda znamo i da je nultočka
jedinstvena.

Traženje nultočki do na zadanu točnost sastoji se od dvije faze:
\begin{enumerate}
    \item Određivanje što manjeg segmenta u kojem se nalazi nultočka funkcije
    \item Nalaženje nultočke iterativnom metodom do na traženu točnost
\end{enumerate}

Osnovni korak numeričkih metoda je izolirati jedan korijen (nultočku, rješenje)
$\xi$ jednadžbe $f(x) = 0$. Nakon što se izolira jedan korijen jednadžbe nekom
odabranom numeričkom metodom formiramo niz aproksimacija $x_i \to \xi$.
Numeričku metodu zaustavljamo kad se zadovolji zadani kriterij točnosti
$$
    |\xi - x_i | < \varepsilon\,.
$$

Vrlo često nije jednostavno prepoznati kad smo dovoljno blizu korijena, ali iz
Teorema o srednjoj vrijednosti slijedi
$$
    |\xi - x_i| = \left| \frac{f(x_i)}{f'(c)} \right|, c \in [\xi, x_i]\,.
$$

\begin{definition}[red konvergencije]
    Neka niz $(x_i)_{i\in\mathbb{N}}$ dobiven nekom iterativnom metodom
    konvergira prema $\xi$. Ako postoje dvije pozitivne konstante $C$, $r$,
    takve da vrijedi
    $$
        \lim_{i\to\infty} \left|\xi-x_{i+1}\right| \leq C|\xi-x_i|^r\,,
    $$
    tada kažemo da \textbf{metoda ima red konvergencije} $r$. 
\end{definition}

\subsection{Metoda bisekcije (polovljenja)}

Metoda bisekcije ili polovljenja ima zanimljivu primjenu za traženje korijena
nelinearnih jednadžbi iako se vrlo često koristi kod pretraživanja monotonih
nizova.

Metoda se oslanja na ideju sukcesivnog ograđivanja rješenja. Segment $[a, b]$ za
koji vrijedi nejednakost $f(a)f(b) \leq 0$ prepolovimo i provjerimo u kojem od
dva novonastala segmenta funkcija $f$ mijenja predznak. Polovljenje segmenta
nastavljamo dok duljina segmenta ne postane dovoljno mala da se zadovolji uvjet
konvergencije.

Metoda bisekcije \textbf{ne može odrediti višestruke nultočke parnog reda} jer u
tim slučajevima funkcija ne mijenja predznak u okolini nultočke.

Metoda bisekcije \textbf{ima najlošiju ocjenu greške} (\textbf{najsporija}
metoda), ali je i metoda koja \textbf{ima sigurnu konvergenciju}.

% TODO: Dodati ilustraciju primjene

\subsubsection{Ocjena greške}

Greška metode se može procjeniti a priori.

$$
    |\xi - x_i| \leq \frac{b_i-a_i}{2} = \dots = \frac{b_0 - a_0}{2^{i+1}}\,.
$$

Ako je željena točnost $\varepsilon$, tada se unaprijed može izračunati potreban
broj iteracija iz nejednadžbe
$$
    \frac{b_0 - a_0}{2^{i+1}} \leq \varepsilon\,.
$$

\begin{example}
    Odredi potreban broj iteracija metode bisekcije kako bi odredila nultočka za
    funkciju $f(x) = x^3 - 2x - 5$ na segmentu $[1,3]$ sa točnošću $\varepsilon
    = 10^{-3}$ .
\end{example}

Određujemo broj iteracija iz nejednadžbe
$$
    \frac{b_0 - a_0}{2^{i+1}} \leq \varepsilon\,.
$$
U ovom slučaju $a_0 = 1$, $b_0 = 3$ pa je
$$
    \frac{3 - 1}{2^{i+1}} \leq 10^{-3} \implies 2^{i+1} \geq 2 \cdot 10^3 \implies
    (i + 1) \log 2 \geq \log (2 \cdot 10^3) \implies i \geq \frac{\log (2 \cdot 10^3)}{\log 2} - 1 \implies i \geq 10\,.
$$
Dakle, potrebno je napraviti najmanje 11 iteracija.

\subsubsection{Brzina konvergencije}

\begin{gather*}
    \xi - x_i \leq \frac{b_i-a_i}{2} = \varepsilon_i\,,\\
    \varepsilon_i={1\over 2} \varepsilon_{i-1}\,.
\end{gather*}

Dakle, budući da je $C={1\over 2}$, $r=1$ bisekcija ima red konvergencije 1,
odnosno metoda bisekcije \textbf{konvergira linearno}.

\subsubsection{Algoritam}

\begin{algorithmic}
\Function{Bisekcija}{$f,\;a,\;b,\;\varepsilon$}
    \State $c \gets \dfrac{a + b}{2}$
    \If{$|b - a| < \varepsilon$}
        \State \Return $c$
    \EndIf
    \If{$f(a)f(c) \leq 0$}
        \State \Return \Call{Bisekcija}{$f, a, c, \varepsilon$}
    \Else
        \State \Return \Call{Bisekcija}{$f, c, b, \varepsilon$}
    \EndIf
\EndFunction
\end{algorithmic}
