\section{Metoda bisekcije (polovljenja)}

Metoda bisekcije ili polovljenja ima zanimljivu primjenu za traženje korijena
nelinearnih jednadžbi iako se vrlo često koristi kod pretraživanja monotonih
nizova.

Metoda se oslanja na ideju sukcesivnog ograđivanja rješenja. Segment $[a, b]$ za
koji vrijedi nejednakost $f(a)f(b) \leq 0$ prepolovimo i provjerimo u kojem od
dva novonastala segmenta funkcija $f$ mijenja predznak. Polovljenje segmenta
nastavljamo dok duljina segmenta ne postane dovoljno mala da se zadovolji uvjet
konvergencije.

Metoda bisekcije \textbf{ne može odrediti višestruke nultočke parnog reda} jer u
tim slučajevima funkcija ne mijenja predznak u okolini nultočke.

Metoda bisekcije \textbf{ima najlošiju ocjenu greške} (\textbf{najsporija}
metoda), ali je i metoda koja \textbf{ima sigurnu konvergenciju}.

% TODO: Dodati ilustraciju primjene

\subsection{Ocjena greške}

Greška metode se može procjeniti a priori.

$$
    |\xi - x_i| \leq \frac{b_i-a_i}{2} = \dots = \frac{b_0 - a_0}{2^{i+1}}\,.
$$

Ako je željena točnost $\varepsilon$, tada se unaprijed može izračunati potreban
broj iteracija iz nejednadžbe
$$
    \frac{b_0 - a_0}{2^{i+1}} \leq \varepsilon\,.
$$

\begin{example}
    Odredi potreban broj iteracija metode bisekcije kako bi odredila nultočka za
    funkciju $f(x) = x^3 - 2x - 5$ na segmentu $[1,3]$ sa točnošću $\varepsilon
    = 10^{-3}$ .
\end{example}

Određujemo broj iteracija iz nejednadžbe
$$
    \frac{b_0 - a_0}{2^{i+1}} \leq \varepsilon\,.
$$
U ovom slučaju $a_0 = 1$, $b_0 = 3$ pa je
$$
    \frac{3 - 1}{2^{i+1}} \leq 10^{-3} \implies 2^{i+1} \geq 2 \cdot 10^3 \implies
    (i + 1) \log 2 \geq \log (2 \cdot 10^3) \implies i \geq \frac{\log (2 \cdot 10^3)}{\log 2} - 1 \implies i \geq 10\,.
$$
Dakle, potrebno je napraviti najmanje 11 iteracija.

\subsection{Brzina konvergencije}

\begin{gather*}
    \xi - x_i \leq \frac{b_i-a_i}{2} = \varepsilon_i\,,\\
    \varepsilon_i={1\over 2} \varepsilon_{i-1}\,.
\end{gather*}

Dakle, budući da je $C={1\over 2}$, $r=1$ bisekcija ima red konvergencije 1,
odnosno metoda bisekcije \textbf{konvergira linearno}.

\subsection{Algoritam}

\begin{algorithmic}
\Function{Bisekcija}{$f,\;a,\;b,\;\varepsilon$}
    \State $c \gets \dfrac{a + b}{2}$
    \If{$|b - a| < \varepsilon$}
        \State \Return $c$
    \EndIf
    \If{$f(a)f(c) \leq 0$}
        \State \Return \Call{Bisekcija}{$f, a, c, \varepsilon$}
    \Else
        \State \Return \Call{Bisekcija}{$f, c, b, \varepsilon$}
    \EndIf
\EndFunction
\end{algorithmic}
